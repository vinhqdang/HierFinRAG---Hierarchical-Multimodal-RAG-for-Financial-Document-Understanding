\section{Related Work}

Current approaches to financial document understanding generally fall into three categories: standard dense retrieval, graph-based RAG, and agentic code generation. Table \ref{tab:comparison} summarizes the key differences between HierFinRAG and these existing paradigms.

\begin{table}[h]
\centering
\resizebox{\textwidth}{!}{%
\begin{tabular}{l c c c c}
\hline
\textbf{Feature} & \textbf{Vanilla RAG} & \textbf{GraphRAG (Generic)} & \textbf{Agentic RAG (GPT-4o)} & \textbf{HierFinRAG (Ours)} \\
\hline
\textbf{Structure Awareness} & Low (Chunks) & Medium (Entities) & High (Code) & \textbf{High (Accounting)} \\
\textbf{Table Representation} & Flattened Text & Knowledge Triples & DataFrame & \textbf{Heterogeneous Subgraph} \\
\textbf{Reasoning Mechanism} & Neural Only & Neural Only & Code Execution & \textbf{Symbolic-Neural Fusion} \\
\textbf{Numerical Precision} & Low & Low & High & \textbf{High} \\
\textbf{Auditability} & Low & Low & Medium & \textbf{High (Hard Routing)} \\
\textbf{Latency} & Low ($<$1s) & Medium & Very High ($>$10s) & \textbf{Medium ($\sim$4s)} \\
\hline
\end{tabular}
}
\caption{Comparison of HierFinRAG against state-of-the-art paradigms. HierFinRAG is unique in its dedicated modeling of financial accounting structures and its deterministic separation of calculation from generation.}
\label{tab:comparison}
\end{table}

While Agentic RAG (e.g., using Python environments) achieves high precision, its latency prohibits real-time use. Generic GraphRAG methods focus on entity relations (Organization-Person) rather than the rigid hierarchical/mathematical relationships found in financial statements. HierFinRAG fills this gap.
